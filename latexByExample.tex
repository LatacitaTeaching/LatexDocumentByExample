\documentclass{article}

\usepackage[utf8]{inputenc}

\title{Mi primer documento \LaTeX}

\author{Pablo Sánchez}

\date{04 de Noviembre de 2019}
 
\begin{document}

\maketitle

\section{Introducción}

Lorem ipsum dolor sit amet, consectetur adipiscing elit. Nulla velit mauris, tempor posuere pharetra nec, mattis eget nunc. Aliquam erat volutpat. Pellentesque porta, dolor a egestas porta, mauris lorem vestibulum augue, sit amet luctus diam risus non purus. Aliquam ac tellus porttitor leo rutrum malesuada. Vestibulum cursus tincidunt dolor, vitae commodo risus mattis nec. Nam viverra aliquet lacus, vitae venenatis mauris tincidunt vel. Quisque sed congue velit, in facilisis dolor. Nullam dapibus massa augue, et fringilla elit pharetra maximus. Maecenas eleifend, velit a pulvinar eleifend, nunc neque feugiat odio, ut condimentum felis tellus sed enim. Aenean enim mauris, fermentum non varius eget, faucibus vitae est. Sed nec tempus massa. Morbi maximus mauris in justo egestas luctus.

Nullam eget varius neque. Mauris feugiat libero id molestie mollis. Morbi mattis efficitur libero, non pharetra ligula maximus vel. Fusce elementum sagittis elit. Proin porta dui at sodales auctor. Praesent hendrerit, nunc non vestibulum maximus, purus lacus ultricies leo, ut venenatis nisl lectus vel leo. Suspendisse vel bibendum magna. Aenean pretium nibh at faucibus cursus. Nunc sem nulla, bibendum sed sagittis in, posuere non tellus. Aenean ullamcorper ornare nibh quis mattis. Ut porttitor, enim sed molestie sodales, dui turpis tempor mauris, ut tristique ipsum nulla a nunc.

\section{Elementos Básicos}

Aenean nec massa non nunc rutrum vestibulum a et turpis. Vestibulum maximus volutpat metus, in cursus mi mattis id. Morbi et eros risus. Suspendisse a feugiat tortor. In at dictum nisl, luctus aliquam dui. Vestibulum placerat ex at turpis interdum, ut tincidunt sapien tincidunt. Mauris dignissim orci eget rhoncus condimentum. Donec efficitur imperdiet nibh, cursus malesuada odio iaculis in. Pellentesque sed odio blandit leo luctus dignissim. Maecenas at libero in elit consequat porttitor pretium eget enim. Aenean posuere arcu turpis, eu pellentesque felis lobortis ac. Nulla risus eros, interdum convallis pellentesque nec, accumsan et tortor. Praesent iaculis placerat eleifend. Etiam sem est, rutrum et leo vitae, euismod varius ligula. Duis bibendum sollicitudin iaculis. Duis tristique finibus tellus, ut sollicitudin mi ullamcorper vel.

\subsection{Secciones y Niveles}
\label{subsec:secciones}

Maecenas et risus fringilla, viverra felis placerat, malesuada elit. Vestibulum lobortis consectetur nisi non convallis. Duis consequat suscipit arcu sed pharetra. Donec lobortis consectetur neque id maximus. In vehicula lorem vitae nisl aliquet, sit amet gravida urna venenatis. Quisque eros lorem, tristique quis arcu sed, lobortis rhoncus velit. Etiam risus ligula, aliquam sit amet dui ac, dictum pellentesque felis. Sed dictum pellentesque rhoncus. Interdum et malesuada fames ac ante ipsum primis in faucibus. Curabitur sed enim sit amet nisi consectetur viverra ut et libero. Duis feugiat ante ac imperdiet bibendum. Morbi posuere, ante sed cursus ultricies, nisi ipsum fermentum justo, non finibus arcu odio at massa. Mauris nec rhoncus erat, et vestibulum diam. Cras sit amet quam mi. Mauris non semper metus.

\subsection{Adornos Básicos}

Con la etiqueta \texttt{emph} podemos decir qué un cierto texto debe quedar resaltado conforme a las \emph{normas de estilo} que se estén aplicando en ese momento.

\subsection{Listas}

En esta sección se muestran diferentes tipos de listas. Para crear una lista:

\begin{enumerate}
	\item Escogemos el tipo de lista a crear.
	\item Escribimos \texttt{begin} más el tipo de lista a crear. 
	\item Cada elemento de la lista se escribe precedido de la etiqueta \texttt{item}.
\end{enumerate}

En \LaTeX hay tres tipos básicos de listas: 

\begin{itemize}
	\item Enumeradas.
	\item Sin enumerar. 
	\item De descripción.
\end{itemize}

Recuerda además que en \LaTeX se puede generar la salida en tres tipos de formatos:

\begin{description}
	\item[DVI] Formato original de \LaTeX, independiente de la plataforma, describe un documento como una secuencia de comandos a dibujar en una salida.
	\item[PS] Alternativa a DVI, pensada para ser un lenguaje procesable por impresoras.   
	\item[PDF] Simplificación de PS, muy orientada al intercambio de documentos electrónicos entre diferentes plataformas.
\end{description}

\section{Figuras, Tablas y Listados de Código}

En la Sección~\ref{subsec:secciones} se explicó como crear secciones. Ahora acabamos de ver cómo referenciarlas adecuadamente. Luego veremos cómo se pueden referenciar figuras, tablas y listados de código.

\subsection{Figuras}

Vivamus id urna sit amet risus pretium vulputate. Fusce scelerisque ex sapien, quis eleifend enim aliquet faucibus. In lorem felis, tincidunt id odio luctus, ullamcorper molestie dolor. Praesent facilisis nisl est, tincidunt iaculis elit convallis id. Praesent interdum magna mi. Duis sit amet dictum quam, ut tempus urna. Fusce ornare semper purus in tempor. Praesent at accumsan nulla, in scelerisque mauris. Mauris blandit mi vestibulum tortor fringilla, at ultricies dui mattis. Nunc posuere venenatis elit vel malesuada. Vestibulum finibus urna est, vitae convallis purus facilisis quis. Sed a sem eu orci cursus congue. Donec sapien sapien, fermentum ac justo et, hendrerit imperdiet purus. Phasellus consequat mauris in scelerisque gravida. Suspendisse sollicitudin porta finibus. Nullam sit amet justo ac sem imperdiet aliquam id vel arcu.

\subsection{Tablas}

Proin eleifend est at accumsan congue. Maecenas eu iaculis leo. Aliquam fermentum dolor quis odio gravida, ut egestas quam porttitor. Aliquam et lacus sodales, eleifend purus vel, sagittis sapien. Donec blandit cursus nibh, a mattis lectus laoreet nec. Aliquam aliquet dolor at facilisis semper. In cursus odio quis nulla vehicula, ac ullamcorper magna pharetra. Nam quis nulla neque. Duis non enim porttitor, malesuada nisi non, tincidunt justo. Cras molestie ornare ante id luctus. Phasellus consectetur ex sed pulvinar ultricies. Sed scelerisque turpis at lorem aliquet, sit amet pharetra nibh mattis. Integer in pretium nunc, id interdum dui. Mauris condimentum mauris in rutrum semper. Aliquam id ante mi.

\subsection{Listados de Código}

Pellentesque malesuada, urna sed consequat tempus, leo dolor cursus turpis, nec tincidunt magna nisi scelerisque nisi. Ut laoreet eget lectus a dictum. Lorem ipsum dolor sit amet, consectetur adipiscing elit. Duis elementum vel lectus iaculis tristique. Quisque mattis pharetra augue a finibus. Orci varius natoque penatibus et magnis dis parturient montes, nascetur ridiculus mus. Ut at varius ipsum. Aliquam vehicula mi ac eros pulvinar lobortis.

\section*{Apéndice. Otras Cuestiones}

\paragraph{Primer punto}

Vestibulum pulvinar sapien eget tellus vestibulum, vitae luctus sapien malesuada. Nulla eu lectus sit amet ante euismod accumsan. Nam luctus eu urna id ultricies. Praesent et ante at mi tempus consectetur. Suspendisse suscipit, lacus id vehicula consequat, dui dolor maximus elit, vitae interdum tellus urna nec odio. Praesent a lacinia velit, ac iaculis enim. \subparagraph {Roma} Nam ac consequat purus, at fringilla est. Sed sollicitudin varius tortor sed lacinia. Integer sagittis purus id lectus auctor pharetra. \subparagraph {Venecia} Curabitur molestie, metus in cursus gravida, libero enim vulputate risus, eget venenatis justo quam sit amet diam.

\paragraph{Segundo punto}

Aliquam pretium dui ut ante sagittis, facilisis velit fringilla. Sed tincidunt sit amet urna non vestibulum. Vivamus sed tellus dignissim felis elementum vehicula venenatis eu ex. Quisque eget justo augue. Interdum et malesuada fames ac ante ipsum primis in faucibus. Aliquam fermentum maximus ex, at pretium lectus ornare vitae. Nullam imperdiet vitae tellus eget sodales. Mauris maximus viverra velit, a aliquam orci varius eget.

\end{document}