Aenean nec massa non nunc rutrum vestibulum a et turpis. Vestibulum maximus volutpat metus, in cursus mi mattis id. Morbi et eros risus. Suspendisse a feugiat tortor. In at dictum nisl, luctus aliquam dui. Vestibulum placerat ex at turpis interdum, ut tincidunt sapien tincidunt. Mauris dignissim orci eget rhoncus condimentum. Donec efficitur imperdiet nibh, cursus malesuada odio iaculis in. Pellentesque sed odio blandit leo luctus dignissim. Maecenas at libero in elit consequat porttitor pretium eget enim. Aenean posuere arcu turpis, eu pellentesque felis lobortis ac. Nulla risus eros, interdum convallis pellentesque nec, accumsan et tortor. Praesent iaculis placerat eleifend. Etiam sem est, rutrum et leo vitae, euismod varius ligula. Duis bibendum sollicitudin iaculis. Duis tristique finibus tellus, ut sollicitudin mi ullamcorper vel.

\subsection{Secciones y Niveles}
\label{subsec:secciones}

Maecenas et risus fringilla, viverra felis placerat, malesuada elit. Vestibulum lobortis consectetur nisi non convallis. Duis consequat suscipit arcu sed pharetra. Donec lobortis consectetur neque id maximus. In vehicula lorem vitae nisl aliquet, sit amet gravida urna venenatis. Quisque eros lorem, tristique quis arcu sed, lobortis rhoncus velit. Etiam risus ligula, aliquam sit amet dui ac, dictum pellentesque felis. Sed dictum pellentesque rhoncus. Interdum et malesuada fames ac ante ipsum primis in faucibus. Curabitur sed enim sit amet nisi consectetur viverra ut et libero. Duis feugiat ante ac imperdiet bibendum. Morbi posuere, ante sed cursus ultricies, nisi ipsum fermentum justo, non finibus arcu odio at massa. Mauris nec rhoncus erat, et vestibulum diam. Cras sit amet quam mi. Mauris non semper metus.

\subsection{Adornos Básicos}

Con la etiqueta \texttt{emph} podemos decir qué un cierto texto debe quedar resaltado conforme a las \emph{normas de estilo} que se estén aplicando en ese momento.

\subsection{Listas}

En esta sección se muestran diferentes tipos de listas. Para crear una lista:

\begin{enumerate}
	\item Escogemos el tipo de lista a crear.
	\item Escribimos \texttt{begin} más el tipo de lista a crear. 
	\item Cada elemento de la lista se escribe precedido de la etiqueta \texttt{item}.
\end{enumerate}

En \LaTeX hay tres tipos básicos de listas: 

\begin{itemize}
	\item Enumeradas.
	\item Sin enumerar. 
	\item De descripción.
\end{itemize}

Recuerda además que en \LaTeX se puede generar la salida en tres tipos de formatos:

\begin{description}
	\item[DVI] Formato original de \LaTeX, independiente de la plataforma, describe un documento como una secuencia de comandos a dibujar en una salida.
	\item[PS] Alternativa a DVI, pensada para ser un lenguaje procesable por impresoras.   
	\item[PDF] Simplificación de PS, orientada al intercambio entre diferentes plataformas	de documentos electrónicos 
\end{description}
